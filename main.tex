%documentMetadata{}
\documentclass[draft, a4paper,12pt]{article}

%\usepackage[italian]{babel}
\usepackage{pgfplots}
\usepackage{amsmath}
\usepackage{float}
\usepackage{graphicx} %for the scheme
\usepackage{array}
\usepackage[margin=1in]{geometry}
\pgfplotsset{width=10cm,compat=1.9}

\begin{document}

\parskip=10pt plus 1pt
\parindent=0pt

\begin{center}
    University of Milan

\par{\Large{Optics, Electronics, and Modern Physics Laboratory}}  

November 30, 2024 
\par


\end{center}


\par
\begin{center}
{\Large\textbf{The Michelson Interferometer}}


    
\par\noindent\rule{\textwidth}{0.4pt}

   abstract
    
    
    \par\noindent\rule{\textwidth}{0.4pt}
\end{center}



%Scopo e modalità (in breve) 
%Descrivere sinteticamente ciò che ci si propone di fare 
%nell'esperimento, ossia cosa si vuole misurare, verificare, provare… Cosa 
%si dovrà misurare? Con che strumento? Potete inserire foto o schemi.
\section{Objectives}
%\section*{Objectives}

The purpose of this experiment is to accurately determine

%Raccolta dati misurati sotto forma di tabella ordinata
%Presentare non solo i dati elaborati ma anche quelli grezzi
%(questi ultimi, eventualmente, in appendice).
%Presentare eventuali grafici con gli assi correttamente definiti (unità di 
%misura, cifre significative, identificativo assi).
\section{Procedure}
\subsection{jjj}
how was the error evaluated?

**Purpose and Procedure (Briefly)**  
Provide a concise description of the goals of the experiment: what is being measured, verified, or tested. What needs to be measured? With which instruments? You can include photos or diagrams.

- **Data Collection in Tabular Format**  
  Present not only processed data but also raw data (the latter can be included as an appendix if necessary).  
  Include any relevant graphs with properly defined axes (units of measurement, significant figures, axis labels). Estimate the measurement errors of the quantities measured (justify the method chosen).  

- **Final Value of the Quantity to Determine and Its Uncertainty**  
  Always report the values with their respective uncertainties, ensuring attention to significant figures:  

- **Comparison with Known Data and Evaluation of Discrepancies.**

- **Additional Notes:** Report any problems encountered, final comments, and interpretations.  

**Note:** Do not include theoretical descriptions or detailed apparatus descriptions, as these will be addressed during the exam presentation.

%Stima degli errori delle grandezze misurate (giustificare il metodo scelto)
%Valore finale della grandezza da determinare ed incertezza
%Ricordarsi di dare i valori con i rispettive errori e attenzione alle cifre significative! 
\section{Data Analysis}

a semi-sum is already divided by two

error lower than the resolution of the instrument declared above...


but why? in table 9 I only see three values with much larger uncertainty than the rest of the measurements, while the others seem in line with the previous two configurations.


only because you have a much larger uncertainty on the anti-parallel configuration


%Confronto con i dati noti e valutazione di discrepanze.
\section{Conclusion}
 
The laser wavelength, the refractive index of air, the coherence lengths of white and green light, and the separation of the sodium D-lines are in good agreement with known or expected values. The use of appropriate error analysis methods, combining statistical treatment of repeated measurements with proper error propagation for derived quantities, ensured realistic uncertainty estimates. 
Difficulties primarily arose in accurate fringe counting, especially when large numbers of fringes were involved or when aligning the interferometer to achieve stable interference patterns. The micrometer screw exhibited slight play, though consistently turning it in the same direction mitigated this issue. Despite these challenges, the experimental methods proved robust, and the selected error estimation approaches — statistical for multiple measurements and partial-derivative propagation for single-step calculations accounted for both random and systematic uncertainties.


\end{document}
