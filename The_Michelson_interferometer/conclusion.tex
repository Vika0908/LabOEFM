The Michelson interferometer experiment successfully measured key optical properties, including the wavelength of a red laser beam, the refractive index of air, the coherence lengths of light packets, and the separation between the two sodium D-lines. The results were as follows:

- The red laser wavelength was determined as \(\lambda = (636.0 \pm 19.0)\,\text{nm}\), consistent with the nominal value of \(632.8\,\text{nm}\) within experimental uncertainty.  

- The refractive index of air at atmospheric pressure was calculated as \(n_a = (1.000266 \pm 0.000012)\), demonstrating excellent agreement with twith the accepted value of \(1.00027\). 

- The coherence lengths were measured as \(L_{\text{c,w}} = (13.33 \pm 1.60)\,\mu\text{m}\) for white light and \(L_{\text{c,g}} = (27.00 \pm 0.58)\,\mu\text{m}\) for green light.  

- The separation between the sodium D-lines was found to be \(\Delta \lambda = (6.41 \pm 0.53)\,\text{Å}\), aligning well with the known value of approximately \(6\,\text{Å}\).

While the experiment achieved high precision, potential sources of error included systematic uncertainties in the displacement measurements (\(\Delta x\)) due to alignment imperfections and mechanical limitations of the micrometer screw. Additional variability in the fringe counts (\(N\)) also contributed to the uncertainty. Improvements such as automating the fringe-counting process and refining the micrometer adjustment mechanism could further enhance measurement accuracy.

Overall, the results confirm the Michelson interferometer's capability for precise optical measurements and provide valuable insights into the wave nature of light and its interaction with matter.
