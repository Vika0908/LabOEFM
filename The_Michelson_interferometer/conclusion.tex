 
The laser wavelength, the refractive index of air, the coherence lengths of white and green light, and the separation of the sodium D-lines are in good agreement with known or expected values. The use of appropriate error analysis methods, combining statistical treatment of repeated measurements with proper error propagation for derived quantities, ensured realistic uncertainty estimates. 
Difficulties primarily arose in accurate fringe counting, especially when large numbers of fringes were involved or when aligning the interferometer to achieve stable interference patterns. The micrometer screw exhibited slight play, though consistently turning it in the same direction mitigated this issue. Despite these challenges, the experimental methods proved robust, and the selected error estimation approaches — statistical for multiple measurements and partial-derivative propagation for single-step calculations accounted for both random and systematic uncertainties.
