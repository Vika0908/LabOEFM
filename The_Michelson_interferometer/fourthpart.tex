\subsection{Purpose and Procedure}
 

The final part of the experiment involves replacing the light source with a sodium lamp. The purpose of this experiment was to measure the separation between the two sodium D-lines, which are closely spaced at approximately 6 Å (\(\lambda_1 = 5890 \, \text{Å}\) and \(\lambda_2 = 5896 \, \text{Å}\)). The sodium light source was introduced into the Michelson interferometer, and the interference fringes produced by the two closely spaced wavelengths were analyzed.

A diaphragm and converging lens were placed between the sodium lamp and the beam splitter to filter and focus the light. This setup helped create a sharp interference pattern on the screen. The movable mirror (S3) was adjusted to measure the displacement (\(\Delta x\)) corresponding to one modulation cycle. This displacement was used to calculate the separation of the sodium doublet lines using the formula:

\[
\Delta \lambda = \frac{m \lambda^2}{2 \Delta x}
\]

Where:
 \(\lambda = 5893 \, \text{Å}\) is the average wavelength of the sodium D-lines,
 \(m\) is the number of sharp interference pattern transitions observed (for this experiment, \(m = 5\)),
 \(\Delta x\) is the displacement measured during the experiment.
\subsection{Analysis and Error Evaluation}

The measured values of \(x_0\), \(x\), and \(\Delta x\) for the separation of the sodium D-lines: 

\begin{table}[!htbp]
    {\par\centering
    \begin{tabular}{cccc}
        \hline
        Measure & $ x_0 \text{ (µm)}$ & $x \text{ (µm)}$ & $\Delta x$ \text{(mm)}\\
        \hline
        1   &   8060& 3790&   0.85 \\
        2   &   6600& 13650&  1.41 \\
        3   &   2300& 10700&  1.68 \\
        4   &   5180& 12110&  1.39 \\
        5   &   6620& 13610&  1.40 \\
        6   &   8090& 15070&  1.40 \\
        \hline
    \end{tabular}
    \par}
    \caption{Measurement of the Separation of Sodium Doublet Lines }
\end{table}

From these measurements, the average mirror displacement is
\[
\overline{\Delta x} = (1.355 \pm 0.111)\,\text{mm}
\]
\(\sigma_{\Delta x}\) was calculated using the standard deviation of the mean method from individual displacement measurements.
In formula for \(\Delta \lambda\), \noindent
$m$ is an integer count of modulation cycles (assumed exact), and 
$\lambda$ is taken from the literature with negligible uncertainty relative to that of $\Delta x$. Therefore, the main source of uncertainty in 
$\Delta\lambda$ comes from \overline{\Delta x}.

\medskip

To propagate the uncertainty, we note that

\[
\varepsilon_{\Delta \lambda}^2 = \left(\frac{\partial \Delta \lambda}{\partial \Delta x}\varepsilon_{\Delta x}\right)^2
\] leading to
\[
\varepsilon_{\Delta \lambda} = \left|-\frac{m \lambda^2}{2 \Delta x^2}\right|\varepsilon_{\Delta x} = \frac{m \lambda^2}{2 \Delta x}\frac{\varepsilon_{\Delta x}}{\Delta x}=\Delta \lambda \frac{\varepsilon_{\Delta x}}{\Delta x}
\]


Since \(\Delta \lambda = 6.41\, \text{Å}\) and \(\varepsilon_{\Delta x}/\Delta x = 0.082\), the absolute uncertainty is

\[
\varepsilon_{\Delta \lambda} =  6.41\,\text{Å} \times 0.082 \approx 0.53\,\text{Å}
\]

The final result for separation of the sodium D-lines is

\[
\Delta \lambda = (6.41 \pm 0.53)\,\text{Å}
\]

This result is in good agreement with the known separation of about 6 Å between the sodium D-lines. The dominant uncertainty arises from the measurement of \(\Delta x\). By using standard error propagation, the final uncertainty in \(\Delta \lambda\) is realistically estimated.

