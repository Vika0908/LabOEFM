\subsection{Purpose and Procedure}

This part aims to determine the refractive index of air (\(n_a\)) under laboratory conditions using a Michelson interferometer. A cylindrical air chamber of length \(D = (5.0 \pm 0.1) \, \text{cm}\) was placed in one arm of the interferometer. Initially, the chamber was evacuated (\(n_a \approx 1\)), and air was gradually reintroduced. The changes in the optical path length caused interference fringes, which were counted to calculate the refractive index.  

The refractive index was determined using the formula:  

\[
n_a = 1 + \frac{N \lambda}{2D},
\]  

where \(N\) is the number of fringes counted, \(\lambda = (635.7 \pm 5.8) \, \text{nm}\) is the laser wavelength, and \(D\) is the length of the air chamber. The uncertainties in \(n_a\) were evaluated by propagating the errors in \(N\), \(D\), and \(\lambda\).  

\subsection{Analysis and Error Evaluation}


Table 2 presents the fringe counts \((N)\), their uncertainty \((\sigma_{N})\), the calculated refractive indices \((n_a)\), and their propagated uncertainties \((\sigma_{n_a})\). Each refractive index was computed using the formula

\[
n_a = 1 + \frac{N\,\lambda}{2\,D},
\]

where \(N\), \(\lambda\), and \(D\) are measured quantities with uncertainties. The error \(\sigma_{n_a}\) for each entry was obtained via standard error propagation:

\[
(\sigma_{n_a})^2
= \left(\frac{\partial n_a}{\partial N}\right)^2 \bigl(\sigma_N\bigr)^2
+ \left(\frac{\partial n_a}{\partial \lambda}\right)^2 \bigl(\sigma_\lambda\bigr)^2
+ \left(\frac{\partial n_a}{\partial D}\right)^2 \bigl(\sigma_D\bigr)^2,
\]

with
\[
\frac{\partial n_a}{\partial N} = \frac{\lambda}{2\,D} , \quad
\frac{\partial n_a}{\partial \lambda} = \frac{N}{2\,D} , \quad
\frac{\partial n_a}{\partial D} = -\,\frac{N\,\lambda}{2\,D^2}
\]

\begin{table}[!htbp]
    \centering
    \begin{tabular}{ccccc}
        \hline
        \text{Measure} & \({N}\) & \({\sigma_{N}}\) & \({n_a}\) & \({\sigma_{n_a}}\) \\
        \hline
        1 & 43 & 2 & 1.000273 & \(2.602\times10^{-5}\) \\
        2 & 42 & 2 & 1.000267 & \(2.599\times10^{-5}\) \\
        3 & 42 & 2 & 1.000267 & \(2.599\times10^{-5}\) \\
        4 & 41 & 2 & 1.000261 & \(2.597\times10^{-5}\) \\
        5 & 41 & 2 & 1.000261 & \(2.597\times10^{-5}\) \\
        \hline
    \end{tabular}
    \caption{Measurement of the Refractive Index of Air}
   
\end{table}

The final refractive index with the final uncertainty determined using a weighted average.

\[
n_a = (1.000266 \pm 0.000012)
\]

The main source of error was the uncertainty in the fringe count (\(\pm 2\)). Automation of the counting process would likely improve precision. Error propagation was chosen as it properly combines the contributions from \(\sigma_{N}\), \(\sigma_\lambda\), and \(\sigma_D\), thereby accounting for both systematic and random effects.
